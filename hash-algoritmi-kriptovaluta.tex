\documentclass[12pt]{article}
\usepackage[utf8]{inputenc}
\usepackage[croatian]{babel}
\usepackage{graphicx}
\usepackage{geometry}%[margin=0.8in]
\usepackage{amssymb}
\usepackage{setspace}
\usepackage[square,numbers]{natbib}
\usepackage{url}
\usepackage{hyperref}
\bibliographystyle{unsrtnat}

\geometry{
	margin=2cm, 
	lmargin=2.5cm
}

%\renewcommand{\bibsection}{\section{\refname}}
\onehalfspacing


\providecommand\phantomsection{}

\hypersetup{
	colorlinks=true,
	linkcolor=black,
	filecolor=magenta,      
	urlcolor=blue,
}
\urlstyle{same}


\begin{document}
\begin{titlepage}
	\begin{center}
		\vspace*{1cm}
		
		\LARGE
		\textbf{IZBORNI PROJEKT} 
		
		\vspace{0.5cm}
		\LARGE
		\textbf{Usporedba hash algoritama u kriptovalutama} 
		
		\vspace{1.5cm}
		
		\Large
		\textbf{Petra Avsec}
		
		\vfill
		
		%A thesis presented for the degree of\\
		%Doctor of Philosophy
		\textsl{}
		\vspace{0.8cm}
		
		\includegraphics[scale=0.5,keepaspectratio]{slike/riteh}
		
		\Large
		Algoritmi i strukture podataka \\
		Zavod za računarstvo\\
		prof. Kristijan Lenac \\
		Tehnički fakultet Sveučilišta u Rijeci \\
		Kolovoz 2020. 
		
		%Department Name\\
		%University Name\\
		%Country\\
		%Date
		
	\end{center}
\end{titlepage}

\tableofcontents

\pagebreak

\section{Kriptovaluta}
Kriptovaluta je vrsta digitalnog novca, elektronski način razmjene novca i zapisivanja transakcija na računalu. To je sredstvo razmjene koje koristi kriptografiju kako bi stvorilo i osiguralo prijenos novca.\cite{cc-wiki} Za razliku od banaka, kriptovalute su decentralizirani sustav što znači da nema središnje organizacije koja kontrolira vrijednost, provjerava transakcije ili na bilo koji drugi način utječe na razmjenu novca. Kriptovaluta je peer-to-peer sustav razmjene što znači da za svoje funkcioniranje koristi rad svojih korisnika, oni su zaslužni za stvaranje novih novčanih jedinica i prijenos postojećih.\\
Sigurnost i integritet transakcija osigurava zajednica rudara (\textit{miners}) koji svojim računalima ovjeravaju transakcije dodavanjem vremenske oznake i ubacivanjem u tzv. glavnu knjigu svih transakcija (\textit{ledger}). Transakcije su računalno zahtjevne i nepraktične za poništiti ili promijeniti što kriptovalutu čini vrlo sigurnim načinom prijenosa novca. \\
Sustav je siguran dok god većinu čvorova kontroliraju "pošteni" (\textit{honest}) korisnici, a ne napadači (\textit{attackers}). \\
Ispravnost svake jedinice kriptovalute osigurava blockchain.\cite{cc-survey}

\subsection{Pregled kriptovaluta}
\cite{bitcoin}


\subsection{Blockchain}
Sustav koji kriptovalute koriste za spremanje podataka o transakcijama je blockchain, decentralizirani mehanizam pohranjvanja informacija o transakcijama. Blockchain je distribuirana glavna knjiga u kojoj se spremaju i čuvaju podaci o transakcijama tako da svi korisnici imaju svoju kopiju, nema osobe ili organizacije koja ima popis transakcija te ih može mijenjati ili brisati. Sve informacije su vidljive svima koji koriste tu valutu te nije moguća manipulacija podataka. \\
Blockchain je realiziran kao rastuća lista podataka raspoređenih u blokove. Svaki blok je povezan s prethodnim tako što sadrži hash vrijednost tog bloka. Osim hash vrijednosti, blokovi sadrže i: index bloka, hash prethodnog bloka, vremenske oznaku, podatake, koji su u slučaju
kriptovaluta, transakcije. \\
Transakcije su strukturirane u obliku Merkle stabla koje se stvara hashiranjem podataka u više navrata. Prvo se hashiraju same transakcije, zatim upare dva hasha koji se ponovno hashiraju.
Postupak se ponavlja dok ne dobijemo samo jedan hash, hash korijen (\textit{root hash}) ili  merkle korijen. Svaki list je hash transakcije, a svaki čvor hash prethodnih hasheva. Kada su podaci spremljeni u ovakvoj strukturi podataka, lako se može provjeriti da li je transakcija spremljena u tom setu podataka. \\
Potvrda da su transakcije određenog bloka prihvaćeni od strane ostalih čvorova i time ispravni, dobije se kada novi čvorovi počnu koristiti hash tog bloka u potrazi za novim.


\subsubsection{Proof of Work}
Proof of work je mehanizam kontrole pristupa koji koristimo kada želimo ograničiti, ali ne i zabraniti pristup resursu. Ovaj mehanizam od korisnika zahtjeva neku vrstu rada, najčešće procesorko vrijeme, i tako odvraća denial-of-service napade i druge vrste iskorištavanja usluga kao što je spam. Proof of work traži od korisnika da izračuna neku funkciju, tzv. \textit{pricing function}.\\
Kriptovalute primjenjuju proof of work tako što miner mora pronaći nonce vrijednost koja hashirana zajedno sa ostalim parametrima koji ulaze u blok, zadovoljava neke uvjete. Kao primjer se može uzeti kriptovaluta Bitcoin, kod koje hashirana vrijednost mora imati određen broj bitova na početku hasha nula. Što je više nula u tom zahtjevu to je teže naći pripradajući hash. \\
Proof of Work se koristi u blockchainu kako bi se stvorio zapis transakcija koji se ne može lako promijeniti, tj. trebalo bi ponoviti potreban rad za sve transakcije koje su bile nakon te koju bi htjeli promijeniti. Najdulji lanac služi kao dokaz svih događaja u lancu i dokaz najveće potrošene procesorske snage.\cite{proof-of-work}


\subsubsection{Hash funkcija}
Hash funkcije su funkcije koje ulazne podatke proizvoljne dužine sažimaju u izlaz određenog formata i veličine. Idealni algoritmi neće imati kolizija, tj. za svaki ulaz, izlaz algoritma je različit, neovisno o veličini promjena ulaznih podataka.\\
Hash funkcije koje se koriste u blockchainu, kriptovalutama i općenito kriptografiji moraju biti jednostrane što znači da ne možemo lako iz izlaza dobiti odgovarajući ulaz.\\
Svojstva optimalnih hash funkcija:
\begin{itemize}
	\item determinističke su - ista poruka (ulazna vrijednost) uvijek rezultira istim hashom
	%it is deterministic, meaning that the same message always results in the same hash
	\item velikom brzinom računaju hash vrijednost bilo koje poruke
	%it is quick to compute the hash value for any given message
	\item teško je i nepraktično generirati poruku koja daje određeni hash
	%it is infeasible (ne moze se lako izracunati, neprakticno je) to generate a message that yields a given hash value
	\item teško je pronaći dvije različite poruke sa istom hash vrijednosti
	%it is infeasible to find two different messages with the same hash value
	\item mala promjena poruke treba imati veliki utjecat na izalznu hash vrijednost kako se te dvije poruke ne bi mogle povezati na temelju sličnih izlaznih vrijednosti (učinak lavine)\cite{ideal-hash-fun}
	%a small change to a message should change the hash value so extensively that the new hash value appears uncorrelated with the old hash value (avalanche effect)
\end{itemize}
%
Sigurne hash funkcije su otporne na sve vrste kriptoanalitičkih napada:
\begin{itemize}
	\item preimage attack - napad u kojem se nastoji pronaći ulaz određenog hasha ukoliko znamo duljinu ulaza. Brute forceom (napad na koristeći grubu silu, u ovom kontekstu slanje svih mogućih ulaza u hash funkciju dok se ne dobije željeni hash) se ulaz može pronaći u $2^N$ evaluacija, ako je $N$ duljina ulaza
	%(imam poznat neki hash i trazim input koji ce mi kao rezultat dat taj hash)
	\item birthday attack - pokušaj pronalaska dva različita ulaza hash funkcije koji rezultiraju istim izlazom (kolizija) - $2^{L/2}$ evaluacija
	\item collision attack - pronađena su dva različita ulaza čiji su izlazi identični
\end{itemize}

\pagebreak
\section{Pregled hash algoritama}
Puno kriptovaluta, različita svojstva blabla.


\subsection{SHA-256}
Secure Hash Algorithm 2 je set kriptografskih funkcija koje je 2001. objavila NSA. Broj 256 u nazivu označava veličinu izlazne vrijednosti funkcije koja iznosi 256 bitova. Postoji nekoliko različitih verzija ovog algoritma sa različitim veličinama izlaza od 224, 384, 512, 512/224 i 512/256 bitova. \\
% ?? output=digest
SHA je jedan od najkorištenijih algoritama u svijetu, a koriste ga među ostalim kriptovalute Bitcoin, Namecoin, Peercoin, Nxt, MazaCoin, te se koristi i u autentikacijskom procesu Debian softver paketa. \\
SHA-256 je napravljen koristeći Merkle-Damgard strukturu koja je pak napravljena pomoću Davies–Meyer jednostrane kompresijske funkcije. Temeljna ideja iza Davies–Meyer konstrukcije je kompresija bloka teksta u $n$ bitova koristeći enkripcijski algoritam. To se postiže stavljanjem nasumične početne vrijednosti od $n$ bitova kao poruke i korištenjem teksta kao ključa. Time se kao produkt enkripcije dobije blok od $n$ bitova. Na rezultat se zatim još primjeni operacija XOR kako bi se smanjila vjerojatnost stvaranja kolizije. \\
Merkle–Damgård hash funkcija je način stvaranja kriptografske hash funkcije otporne na kolizije iz jednostranih kompresijskih funkcija. Koristi se u SHA algoritmu kod rastavljanja ulaza na blokove duljine 512 bitova, koji se zatim obrađuju jedan po jedan koristeći jednostrane kompresijske funkcije bazirane na Davis-Meyeru.

\subsubsection{Opis rada}
Poruka se dijeli u blokove duljine $512$ bitova, a ukoliko je blok manji, proširimo ga do 512 bitova. Blokovi se zatim obrađuju jedan po jedan.\\
Poruka se proširuje tako da se duljini poruke  nadoda jedan bit na kraj čime se dobije $N$ bitova. Na kraj dobivenog dodaje se $512 - 64 - N$ nula, a u zadnja 64 bita se zapiše duljinu originalne poruke.
Blokovi od $512$ bitova dijele se na $16$ dijelova po $32$ bita ($m0 - m15$).\\

Pseudokod:
\begin{itemize}
	\item 8 početnih hash vrijednosti duljine 32 bita (ostaci korijena prvih osam prim brojeva):
	\begin{itemize}
		\item $H(0)_1 = 6a09e667$
		\item $H(0)_2 = bb67ae85$
		\item $H(0)_3 = 3c6ef372$
		\item $H(0)_4 = a54ff53a$
		\item $H(0)_5 = 510e527f$
		\item $H(0)_6 = 9b05688c$
		\item $H(0)_7 = 1f83d9ab$
		\item $H(0)_8 = 5be0cd19$
	\end{itemize}
	\item Glavna petlja ($N$ iteracija gdje je $N$ broj blokova poruke):
	\begin{enumerate}
		\item inicijalizacija $a, b, c, d, e, f, g, h$ sa vrijednostima $h_1$ do $h_8$
		\item primjena SHA-256 kompresijske funkcije čime se dobiju nove vrijednosti varijabli $a - h$
		\item računanje novih $h_1 - h_8$, tako da $h_1 = a + h_1 . . . h_8 = h + h_8$
	\end{enumerate}
	Rezulat izvršavanja petlje je hash poruke s početka.
	\item Kompresijska funkcija:
	\begin{itemize}
		\item 64 iteracije, 6 logičkih funkcija koje se koriste za dobivanje novih vrijednosti $a - h$. Sve funkcije rade sa 32-bitnimm riječima(?) i izlaz im je velićine 32 bita. U računanju se koriste i 64 konstante, 32-bit riječi $K_0-K_{63}$ (ostatci kubnih korijena prvih 64 prostih brojeva)
		\item u računanju se za kriptiranje koristi: bitwise XOR, bitwise AND, bitwise OR, bitwise complement, zbrajanje mod $2^{32}$, posmak u desno za $n$ bitova, rotacija u desno za $n$ bitova\cite{sha-description}
		\begin{figure}[h!]
			\centering
			\includegraphics[width=\linewidth]{slike/sha256_compression}
			\caption{Kompresijska funkcija SHA-256 algoritma. Slika prikazuje j-ti korak funkcije, a $\boxplus$ označava mod $2^{32}$ zbrajanje}
		\end{figure}
	\end{itemize}
	
	Razina zaštite, sigurnosti, u kriptografiji je mjera snage(?) hash funkcije. Izražena je u bitovima, $n$ - bit security, sto znaci da napadač treba izvesti $2^n$ operacija da bi razbio funkciju.\\ \textit{Broken} hash funkcija - uspjeli smo izvesti barem jedan od napada: collision attack i preimage attack. \\ 
	SHA-256 ima $128$ bitnu razinu zaštite.
\end{itemize}

\pagebreak

\subsection{X11}
X11 algoritam je, isto kao i SHA-256, proof of work algoritam. Za razliku od ostalih algoritama koristi drugačiji pristup hashiranju, ulančavanje algoritama. Razvio ga je Evan Duffield te je algoritam 2014. implementiran u protokol Darkcoin kriptovalute, kasnije preimenovane u DASH.
Napravljen je kako bi se otežala izrada ASIC-a koji će efikasno rudariti X11 algoritam, iako danas više nije ASIC otporan.\\
ASIC (application-specific integrated circuit) IC čip spečificno izrađen za neki zadatak, svrhu. Kod kriptovaluta njihova uloga je računanje hash algoritama kriptovaluta (kao primjer možemo uzeti Bitcoin ASIC miner koji je dizajniran za računanje SHA-256). \\
ASIC mineri se oduvijek smatrju prjetnjom sustava rudarenja i kriptovaluta. Glavni su uzrok centralizacije hashing snage. GPU i CPU su u nepovoljnoj poziciji kraj ASIC-a ako se uzme u obzir rudarenje blokova i dobivanje nagrada za to, te tako ograničavaju većinu potencijalnih korisnika od rudarenja kriptovalute. 
ASICs su superiorniji CPU i GPU zato što mogu izračunati više hasheva po sekundi, tako da su rudari koji koriste ASIC u prednosti nad ostalima.
% (https://coinguides.org/x11-algorithm-coins/)
S obzirom na navedene poteškoće s kojima se susreću kriptovalute te njihovi rudari, X11 algoritam je dizajniran kako bi bio efikasan i na CPU i GPU. Osmišljen je da bi bio čim duže ASIC resistant (ili duže nego ostale konkurentne kriptovalute) kako bi "hobbyists" mogli dulje sudjelovati u rudarenju, što je vrlo povoljno za Dash kriptovalutu, njeno širenje i dobru distribuciju. \\  
-X11 algoritam je kombinacija 11 hash funkcija: BLAKE, BLUE MIDNIGHT WISH (BMW), Grøstl, JH, Keccak, Skein, Luffa, CubeHash, SHAvite-3, SIMD, ECHO. Poruka se predaje prvoj hash funkciji koja ga obradi i proslijedi svoj izlaz sljedećoj funkciji kao poruku.\\
X11 je sigurniji od SHA256. Sve hash funkcije koje se koriste kod X11 algoritma su bile kandidati kod traženja novog standarda, boljeg sigurnijeg algoritma SHA3 koji je temeljen na Keccak funkciji.
Razvijeno je još algoritama koji se temelje na istoj ideji kao i X11: X13, X14, X15 i X17 koji koriste više hash funkcija.\\
Kriptovalute koje koriste X11 su Dash, Hatch, Pura, SmartCoin, CannabisCoin, Influxcoin, StartCoin, Onix, i mnoge druge.\cite{x11-description}


\pagebreak
\subsection{SCRYPT}
Još jedan u nizu hash algoritama koji su razvijeni s namjerom  da se izbjegne stvaranje ASIC-ova te maksimalno oteža njihovo rudarenje kriptovaluta. Scrypt algoritam je vrlo memorijski zahtjevan - osim što zahtjeva od rudara da brzo stvaraju brojeve, ti brojevi se spremaju u RAM i treba im se pristupiti prije dobivanja konačnog hasha. Ovim pristupom se drastično smanjuje učinkovitost integriranih krugova specifične namjene.\\
Scrypt je osmislio Colin Percival za spremanje online sigurnosnih kopija UNIX operativnih sustava. Algoritam dodatno otežava rješavanje dodajući šum (noise) - nasumično generirane brojeve te tako povećava vrijeme potrebno za dobivanje hasha. \\
Koriste ga kriptovalute Litecoin, Dogecoin, ProsperCoin, MonaCoin...\\
Algoritam je sačinjen nekoliko parametara:
\begin{itemize}
	\item N - parametar koji označava koliko CPU/memorije traži algoritam
	\item p - paralelizacijski parametar, pozitivni cijeli broj
	\item r - veličina bloka
	\item S - "sol", nasumična vrijednost koja se često koristi kao dodatni ulaz u kriptografskim funkcijama. Služi kao zaštita od \textit{Rainbow table} napada
	\item P - ulazna vrijednost, niz znakova koji želimo hashirati
	\item \textit{dkLen} - željena duljina izlazne vrijednosti u oktetima
\end{itemize}
%
Navedene podatke prosljiedimo \textit{key derivation} funkciji PBKDF2 (funkcija kojom izvlačimo tajne ključeve iz vrijednosti kao što su master key ili neke vrste lozinke). PBKDF2 je akronim za Password-Based Key Derivation Function 2, kojom smanjujemo ranjivost na napade grubom silom (brute force) tako da dobijemo izvedeni ključ.\cite{scrypt}

\pagebreak
\subsection{Ethash}

\pagebreak
\section{Zaključak}

\phantomsection
\addcontentsline{toc}{section}{\listfigurename}

\pagebreak


\phantomsection
\addcontentsline{toc}{section}{\refname}
\bibliography{kripto}



	
\end{document}